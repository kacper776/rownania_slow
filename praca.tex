\documentclass[leqno, 12pt]{article}
\usepackage[T1]{fontenc}
\usepackage[utf8]{inputenc}
\usepackage[polish, english]{babel}
\usepackage{courier}
\usepackage[colorlinks,citecolor=blue,urlcolor=blue]{hyperref}
\usepackage{tabto}
\usepackage{amssymb, amsmath, amsfonts, amsthm, cite, mathtools, enumerate, rotating, hyperref, dsfont}
\usepackage{geometry}
\newgeometry{vmargin={15mm}, hmargin={20mm,30mm}}
\newcommand{\T}{\top}
\newcommand{\e}{\mathcal{E}}
\newcommand{\cs}{\begin{cases}}
\newcommand{\ecs}{\end{cases}}
\newcommand{\RR}{\mathbb{R}}
\newcommand{\NN}{\mathbb{N}}
\newcommand{\CC}{\mathbb{C}}
\newcommand{\QQ}{\mathbb{Q}}
\newcommand{\ZZ}{\mathbb{Z}}
\newcommand{\TT}{\mathbb{T}}
\newcommand{\II}{\mathbb{I}}
\newcommand{\E}{\mathds{E}}
\newcommand{\Tau}{\mathcal{T}}
\newcommand{\PP}{\mathcal{P}}
\newcommand{\Cont}{\mathfrak{c}}
\newcommand{\eps}{\varepsilon}
\newcommand{\xor}{\oplus}
\newcommand{\nsub}{\triangleleft}
\newcommand{\cont}{\mathfrak{c}}
\newcommand{\F}{\mathcal{F}}
\newcommand{\lst}{\preceq_{ST}}

\title{Implementacja algorytmu dla równań słów}
\author{Kacper Solecki}
\date{}

\begin{document}
\maketitle

\section{Wstęp?}
\section{Opis algorytmu}

\subsection{Ogólny zarys}
Algorytm jako wejście przyjmuje równanie słów, w którym litery są reprezentowane przez małe litery angielskiego alfabetu, a zmienne przez wielkie litery. Rozwiązaniem takiego równania, o ile istnieje,
jest podstawienie (ciąg liter) za każdą zmienną występującą w równaniu, takie, że po zamianie zmiennych na ich podstawienia dwie strony równania rzeczywiście są równe jako słowa.
%W języku gramatyk bezkontekstowych litery to symbole terminalne, zmienne --- nieterminalne. 
%Artur: jakich gramatyk - tu nie ma żadnych gramatyk.

Przykładowo jednym z rozwiązań równania $$abYX = XYba$$ jest
$$
\begin{cases}
X = aba \\
Y = a
\end{cases}
$$

Algorytm przeszukuje wszerz graf, którego wierzchołkami są równania słów. 
Przeszukuje te równania, które zawierają litery i zmienne z początkowego równania oraz których długość (rozumiana jako łączna liczba wystąpień liter i zmiennych) nie przekracza $8n^2 + n$, gdzie $n$ jest długością początkowego równania. Zatem przeszukiwany graf, choć wykładniczej względem wyjściowego równania wielkości, jest skończony.
%[dowód poprawności tego we wstępie?]
%Artur: Cytownaie, że tyle wystarczy.

W każdym wierzchołku wykonywane są dwa rodzaje transformacji, które przekształcają równianie w inne. W języku grafów tłumaczy się to na istnienie dwóch rodzajów krawędzi z każdego wierzchołka:

\subsubsection*{Krawędzie łączenia maksymalnych bloków}
Przez maksymalny blok litery $a$ w równaniu $r$ będziemy rozumieć nierozszerzalny, spójny ciąg litery $a$ w $r$. Nierozszerzalny znaczy taki, że z każdej jego strony jest brzeg równania lub inna litera lub zmienna i jeżeli jest to zmienna, to jej podstawianie nie ma litery $a$ ze strony tego bloku (tzn. nie zaczyna lub nie kończy się na $a$).

Taka krawędź jest pomiędzy wierzchołkami $u$ i $v$, jeżeli z równanie $u$ można przekształcić w równanie $v$ łącząc wszystkie maksymalne bloki jakiejś litery występującej w $u$. Przez złączenie bloku rozumiemy zastąpienie go nową literą, niewystępującą w $u$. Oczywiście bloki tej samej długości powinny zostać złączone w tę samą literę, a różnej długości --- w różne litery.
Różne podstawiania za zmienne mogą implikować różne równości pomiędzy blokami, zatem trzeba będzie rozpatrzeć wszystkie przypadki.


\subsubsection*{Krawędzie łączenia par liter}
Innym rodzajem przekształcenia równania $u$ jest wybranie z niego dwóch liter $a$ i $b$, a następnie zastąpienie każdego wystąpienia pary $ab$ poprzez nową literę spoza $u$.


Przy czym nie chcemy, by jedna litera z pary $ab$ mogła znaleźć się w podstawieniu za jakąś zmienną, a druga nie, tzn. jeżeli $X$ jest zmienną, której podstawianie zaczyna się literą $b$, nie chcemy skracać pary $ab$ jeśli w $u$ jest fragment $aX$ (analogicznie jeśli podstawianie $X$ kończy się na $a$, nie chcemy skracać $ab$, jeśli w $u$ jest fragment $Xb$). W przeciwnym przypadku w dalszym przeszukiwaniu grafu stracilibyśmy informację, że litera do której złączyliśmy $ab$ jest rzeczywiście równa tej parze liter. Aby móc skrócić taką parę $ab$, będziemy musieli dla każdej zmiennej rozważyć przypadki, czy jej podstawianie zaczyna się na $b$ i czy kończy się na $a$.
%Artur: dlaczego nic nie ma o tym przypadku dla bloków?


Przeszukiwanie grafu kończy się, gdy dojdziemy do wierzchołka, który reprezentuje równanie, które w oczywisty sposób jest spełnione np.\ $a=a$. Może też być tak, że po przeszukaniu całego grafu nie znajdziemy rozwiązania --- znaczy to, że żadne nie istnieje. Przeszukiwanie zawsze się kończy, dzięki temu, że algorytm pamięta wierzchołki, które już odwiedził i nie rozpatruje ich drugi raz.



Algorytm zaimplementowany jest w języku Python, jego fragmenty będą się pojawiały w dalszej części pracy.

\subsection{Reprezentacja słów}
Słowo, czyli ciąg liter i zmiennych jest reprezentowane jako lista następujących obiektów:
\begin{verbatim}
class Variable(object):
    def __init__(self, nr: int) -> None:
        self.nr = nr

    def __repr__(self) -> str:
        return f'var_{self.nr}'

    def __eq__(self, other) -> bool:
        return type(self) == type(other) and self.nr == other.nr
    
    def __hash__(self):
        return hash(self.nr)


class Letter(object):
    def __init__(self, nr: int, cnt: int) -> None:
        self.nr = nr
        self.cnt = cnt

    def __repr__(self) -> str:
        return f'{self.cnt}({self.nr})'

    def __eq__(self, other) -> bool:
        return type(self) == type(other) and self.nr == other.nr

    def __hash__(self) -> int:
        return hash((self.nr, self.cnt))
\end{verbatim}
Zmienne posiadają tylko jeden atrybut --- numer, który je identyfikuje. Litery oprócz identyfikującego numeru posiadają atrybut $cnt$, który służy krótszemu zapisowi wielokrotnego, spójnego wystąpienia danej litery. Np., zakładając, że litera $a$ ma numer $97$, słowo $aaa$ będzie skrótowo zapisane, jako pojedynczy obiekt Letter o $nr = 97$ i $cnt = 3$.


\subsection{Reprezentacja podstawień}
W trakcie działania algorytmu w każdym wierzchołku musi on pamiętać wszystkie aktualne podstawiania za zmienne. Oprócz tego, ponieważ przechodząc każdą krawędzią w grafie tworzymy nowe litery, zatem musi też pamiętać ich rozwinięcia, tzn.\ jakim słowom złożonym z literek wyjściowego równania odpowiadają litery obecne w danym wierzchołku. Np.\ po złączeniu pary $ab$ w literkę $c$ musimy pamiętać rozwinięcie $c \rightarrow ab$.
%Artur: ale to jest informacja na krawędzi: to samo równanie można uzyskać jako różne kompresje
%i te same litery mogą odpowiadać różnym parom w różnych równaniach.

Aby uprościć implementację i oszczędzać zasoby, algorytm używa trzech rodzajów rozwinięć liter:

\begin{enumerate}
	\item Rozwinięciem może być jawnie zapisane słowo złożone z liter z wyjściowego równania. Takie rozwinięcia są używane tylko dla tych właśnie liter, więc zawsze mają postać $a \rightarrow a$.\newline (Algorytm pamięta je jako pary $(-1, a)$.)

	\item Rozwinięciem może być konkatenacja dwóch innych rozwinięć --- tak dzieje się, gdy łączymy parę liter w inną literę.\newline (Algorytm pamięta je jako pary $(0, (p1, p2))$, gdzie $p1$ i $p2$ to rozwinięcia.)

	\item Rozwinięciem może być wielokrotne powtórzenie innego rozwinięcia --- tak dzieje się, gdy łączymy maksymalne bloki litery do innych liter.\newline (Algorytm pamięta je jako pary $(cnt, p)$, gdzie $cnt > 0$ jest liczbą powtórzeń, a $p$ jest powtórzeniem.)

\end{enumerate}


Podstawianie za zmienną zawsze będzie ciągiem rozwinięć liter. Podstawiania są reprezentowane jako para list --- jedna oznacza początek podstawienia, druga koniec. Na przykład, gdy stwierdzamy, że zmienna $X$ zaczyna się literą $a$, do listy początkowej podstawienia za $X$ dodajemy rozwinięcie litery $a$.

Zamianę takiej reprezentacji podstawienia w napis (zwykły, pythonowy) realizuje funkcja:
\begin{verbatim}
def read_result(result: tuple) -> str:
    # reads substitution for a variable
    def read_part(part: tuple) -> str:
        cnt, res = part
        if cnt == -1:
            return res
        if cnt == 0:
            return read_part(res[0]) + read_part(res[1])
        return cnt * read_part(res)
    return reduce(lambda a, b: a + b,
                  [read_part(part) for part in result[0]], '')\
           + reduce(lambda a, b: a + b,
                    [read_part(part) for part in reversed(result[1])], '')
\end{verbatim}
Zauważmy, że lista oznaczająca koniec podstawienia musi być odwrócona, ponieważ jeżeli dla zmiennej $X$ najpierw zdecydowaliśmy, że kończy się na $a$, a potem że pozostała część kończy się na $b$, lista końcowa $X$ będzie wyglądała \newline$[\text{<rozwinięcie }a\text{>, <rozwinięcie }b\text{>}]$, natomiast taka kolejność decyzji implikuje, że $X$ kończy się na $ba$.


\subsection{reprezentacja grafu}
Wierzchołki przeszukiwanego grafu są reprezentowane przez obiekt Node, który posiada cztery atrybuty:
\begin{verbatim}
class Node(object):
    def __init__(self, L: list, R: list, 
                 letter_unwrap: dict, variable_subs: dict) -> None:
        self.L = L
        self.R = R
        self.letter_unwrap = letter_unwrap
        self.variable_subs = variable_subs

    (...)
\end{verbatim}
Atrybuty L oraz R oznaczają słowa będące odpowiednio lewą i prawą stroną równania reprezentowanego przez ten wierzchołek. Atrybut letter\_unwrap to słownik zawierający dla każdego numeru litery z równania jej rozwinięcie. variable\_subs to również słownik, który dla każdego numeru zmiennej z równiania trzyma jej aktualne podstawianie.


Chcąc znaleźć dzieci danego wierzchołka algorytm iteruje się po wszystkich podzbiorach zmiennych w równaniu i tworzy dzieci nieposiadające tych zmiennych --- tzn.\ takie, w których podstawienie za te zmienne jest już gotowe i nie będzie rozszerzane. Następnie dla każdego takiego podzbioru szuka wierzchołków do których prowadzą oba rodzaje krawędzi.

\subsubsection*{Dzieci względem krawędzi łączenia maksymalnych bloków}
Algorytm iteruje się po wszystkich literach w równaniu i próbuje łączyć maksymalne bloki tej litery (dla każdej powstanie rozłączny zbiór dzieci).
Aby połączyć maksymalne bloki ustalonej litery $a$ najpierw trzeba te bloki znaleźć. Dla każdej zmiennej $X$ z równania przez $X_L$ i $X_R$ oznaczymy długość spójnego bloku liter $a$, który w szukanym rozwiązaniu jest odpowiednio z lewej i z prawej strony podstawienia za $X$. Wtedy długość każdego bloku liter $a$ można przedstawić jako sumę stałych oraz tak opisanych zmiennych. Np.\ w słowie $abXaaaY$ bloki litery $a$ mają długość $1$, $X_L$, $X_R + 3$, $Y_L + 3$ oraz $Y_R$.

Dalej --- interesuje nas, które z bloków po lewej stronie równania będą odpowiadać którym z prawej strony. Zatem, jeżeli po lewej stronie równania są po kolei bloki długości $l_1, l_2, ...., l_n$, a po prawej $r_1, r_2, ..., r_m$, algorytm sprawdzi wszystkie możliwe zbiory równości pomiędzy blokami z lewej i z prawej, takie, że jeżeli w zbiorze są równości $l_a = r_b$ oraz $l_c = r_d$ i $a < c$ to $b < d$ (gdy $a < c$ nie może być takiej sytuacji, że blok nr $a$ z lewej strony odpowiada blokowi nr $b$ z prawej, a blok nr $c$ z lewej odpowiada blokowi nr $d$ z prawej i jednocześnie $d > b$). Oczywiście taki zbiór równości nie zawsze ma rozwiązanie. Aby sprawdzać istnienie rozwiązania takiego zbioru równać diofantycznych, oraz znajdować te rozwiązania użyta została zewnętrzna biblioteka Z3 stworzona przez Microsoft Research.

\subsubsection*{Dzieci względem krawędzi łączenia par liter}
Algorytm iteruje się po wszystkich parach $(a, b)$ różnych liter z równania i próbuje złączyć parę $ab$.
Dla danej pary $ab$ dla każdej zmiennej z równania sprawdzane są wszystkie możliwości tego, czy w szukanym rozwiązaniu zmienna zaczyna się na $b$ i czy kończy się na $a$. Dla każdej takiej możliwości, dla każdej zmiennej $X$, jeżeli zaczyna się ona na $b$ jest zastępowana przez $bX$, a początek jej aktualnego podstawienia jest wydłużany o $b$. Analogicznie, jeżeli zaczyna się ona na $a$ jest zastępowana przez $Xa$, a koniec jej aktualnego podstawienia jest wydłużany o $a$. Następnie każda para $ab$ występująca w równaniu zastępowana jest przez nową literę $c$, spoza równania.

Dla niektórych równań można bardzo łatwo znaleźć równoważne równanie, które jest krótsze. Jeżeli strony równania mają wspólny prefiks (lub wspólny prefiks), można go usunąć. Przykładowo równanie $aXYba = aXXYa$ można uprościć do $Yb = XY$. Za każdym razem, gdy algorytm tworzy dziecko wierzchołka, upraszcza jego równanie w ten sposób, aby zredukować liczbę wierzchołków do przeszukania.

\subsection{Przeszukiwanie grafu}
Graf trawersowany jest za pomocą przeszukiwania wszerz --- takim sposobem, mimo, że cały graf ma długość wykładniczą względem długości początkowego równania, możemy po nim przechodzić i robić jakikolwiek postęp.

\begin{verbatim}
def bfs(start: Node, max_len: int) -> dict:
    # breadth first search
    if start.terminal():
        return start.variable_subs
    q = Queue()
    q.put(start)
    visited = {hash(start)}
    while not q.empty():
        node = q.get()
        for child in node.children():
            child_hash = hash(child)
            if child_hash not in visited and len(child) <= max_len:
                visited.add(child_hash)
                if child.terminal():
                    return child.variable_subs
                q.put(child)
    return {}
\end{verbatim}

Metoda Node.terminal() sprawdza, czy wierzchołek reprezentuje trywialnie spełnione równanie, w którym każde podstawianie za zmienną jest niepuste:

\begin{verbatim}
class Node(object):
    (...)
    def terminal(self) -> bool:
        # checks if node represents correct solution
        for substitution in self.variable_subs.values():
            if not substitution[0] and not substitution[1]:
                return False

        if all(map(is_variable, self.L + self.R)):
            left_side = reduce(lambda a, b:
                                 a + read_result(self.variable_subs[b.nr]),
                               self.L, '')
            right_side = reduce(lambda a, b:
                                  a + read_result(self.variable_subs[b.nr]),
                                self.R, '')
            return left_side == right_side

        return not self.L and not self.R
    (...)
\end{verbatim}

Są dwie możliwości, kiedy równanie uznajemy za trywialnie spełnione

\begin{enumerate}
	\item Równanie jest puste. Ponieważ w każdym kroku upraszczamy równania, każde równanie postaci $w=w$, gdzie $w$ to dowolne słowo, natychmiast upraszcza się do pustego.

\item Równanie zawiera tylko zmienne, a po podstawieniu za nie aktualnych podstawień otrzymujemy równanie postaci $w=w$.
\end{enumerate}

Dodatkową optymalizacją w przeszukiwaniu grafu jest implementacja metody Node.\_\_hash\_\_(), dzięki której za takie same uznawane są wierzchołki zawierające równania takie same co do zamiany numerów zmiennych --- takie równania nazwiemy izomorficznymi.

\begin{verbatim}
class Node(object):
    (...)
    def __hash__(self) -> int:
        rep_L, rep_R = representative_eq(self.L, self.R)
        return hash((tuple(rep_L), tuple(rep_R)))

    (...)


def representative_eq(L: list, R: list) -> Node:
    # for given equation build represantative node of it
    def translate(side: list, letter_nr_trans: dict, free_nr: int):
        result = []
        for symbol in side:
            if is_variable(symbol):
                result.append(copy(symbol))
            else:
                if symbol.nr in letter_nr_trans:
                    result.append(Letter(letter_nr_trans[symbol.nr], symbol.cnt))
                else:
                    letter_nr_trans[free_nr] = symbol.nr
                    result.append(Letter(free_nr, symbol.cnt))
                    free_nr += 1
        return result, free_nr

    letter_nr_trans = {}
    rep_L, free_nr = translate(L, letter_nr_trans, 0)
    rep_R, _ = translate(R, letter_nr_trans, free_nr)
    return rep_L, rep_R
\end{verbatim}

Funkcja representative\_eq() zwraca równanie izomorficzne z danym równaniem, taki że spośród wszystkich takich, jego ciąg numerów kolejnych wystąpień liter jest najmniejszy leksykograficznie. 

\end{document}